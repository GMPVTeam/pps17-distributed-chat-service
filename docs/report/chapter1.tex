\chapter{Processo di Sviluppo}
\section{Metodologia di sviluppo}
Relativamente al processo di sviluppo è stato deciso di adottare e testare una
metodologia \textbf{Agile Scrum-like}. \newline
Nella versione da noi utilizzata
i membri del team sono allo stesso livello, non esiste infatti
uno \textit{Scrum Leader}, ne uno \textit{Scrum Manager}. Il ruolo del \textit{Product Manager} è
stato ricoperto da \textit{Nicola Piscaglia}. 
Come alternativa ad una vera e propria \textit{Kanban} abbiamo
deciso di usare \textbf{Trello}: qui, come fasi dello sprint abbiamo usato \textit{"Backlog"}, \textit{"ToDo"}, \textit{"Design"}, 
\textit{"Develop"}, \textit{"Review"} e \textit{"Done"}. Abbiamo reputato corretto fare
sprints con cadenza settimanale, poichè così è possibile ottenere una buona quantità di feedback
dall'utente e numerose opportunità per imparare e migliorare. \newline
All'inizio del processo di sviluppo abbiamo delineato il \textit{Product Backlog}, contenente gli items, 
con relativa priorità e dimensione stimata; inoltre, è stato fatto un primo sforzo per la modellazione del sistema
in termini di servizi e patterns architetturali, producendo: un diagramma relativo alla gerarchia dei servizi,
uno schema relazionale ed uno logico dei databases.
\newline
All' inizio di ogni settimana abbiamo fatto un meeting, tutti assieme,
per decidere quali elementi del \textit{Product Backlog} assegnare al
prossimo sprint, strutturando lo
\textit{Sprint Backlog}, delineando con più precisione la dimensione degli items. 
\\A seguito di ciò,
i membri del team hanno scelto gli items seguendo le priorità già delineate 
ed hanno iniziato a fare il design ed implementarne le features descritte tramite \textit{TDD}, senza più cambiare le decisioni
 relative agli items prese all'inizio dello sprint. 
 \\\textit{Il Test Driven Development} è un modello di sviluppo che prevede che la stesura dei tests sia fatta prima dell'implementazione e che lo sviluppo sia orientato esclusivamente all'obiettivo di passare i tests precedentemente predisposti. Più in dettaglio, \textit{TDD} prevede un breve ciclo di sviluppo in tre fasi: nella prima si scrive un test per la nuova funzione da sviluppare, nella seconda si sviluppa il codice necessario per passare il test; infine, nella terza si esegue il refactoring del codice per adeguarlo ad determinati standards di qualità.
\newline
Durante lo sviluppo, ogni singolo elemento del team ha la possibilità di fare \textit{Pull Requests} al 
\textit{Repo verità} (non si può fare push): a quel punto,
un altra persona del team controllerà la qualità del codice della singola \textit{PR} e, 
se tutto risulta a norma, farà merge nel branch dello sprint corrente.
Lo strumento di \textit{continuous integration} è stato configurato per permettere il merge di una pull request solo in caso di procedura di build andata a buon fine e solo se tutti i tests terminano con successo; Infine, è stato impostato lo stesso strumento in modo da
generare in automatico l'eseguibile del progetto.
\newline
Al termine di ogni \textit{Sprint} i componenti del team hanno preso parte ad un meeting, in cui sono state
poste in atto le fasi di \textit{Sprint Review} (per valutare il risultato dello Sprint e stabilire le priorit`a delle
prossime feature da portare a termine) e di \textit{Retrospective} (per analizzare il processo di sviluppo
allo scopo di migliorare la produttività del team).
\\
Si è reso necessario, a volte, contattare elementi del team tramite Skype o strumenti simili, 
senza riunire però tutti gli elementi, per risolvere problemi o per procedere con lo sviluppo.
Infine, gli items non completati entro la fine dello sprint sono stati semplicemente riportati
nello sprint successivo.

\section{Strumenti adottati}
I principali strumenti utilizzati durante l'intero processo di sviluppo sono i seguenti:
\begin{itemize}
\item \textbf{Git}: come sistema di versioning abbiamo optato per Git per la sua diffusione, velocità e semplicità di utilizzo;
\item \textbf{GitHub}: è stato scelto GitHub come servizio di hosting;
per il version control e per il repository, poichè permette ampia possibilità di integrazione con altri strumenti di sviluppo. Il repository è disponibile al seguente \href{https://github.com/GMPVTeam/pps17-distributed-chat-service}{link};
\item \textbf{GitFlow}: come metodologia di \textit{branching} all’interno del repository abbiamo adottato
GitFlow, poichè rende lo sviluppo parallelo molto semplice;
\item \textbf{Google Sheets}: la creazione e gestione del \textit{Product Backlog}, \textit{Sprint Backlog}, \textit{Sprint Review} e \textit{Sprint Retrospective} sono state fatte tramite Google Sheets, vista la sua semplicità di utilizzo. È possibile visualizzare i documenti predetti al seguente \href{https://docs.google.com/spreadsheets/d/13H6Xl3ZzywzSpik3gxQbl6CnDtzObbCIdqROZq82Zk4/edit?usp=sharing}{link};
\item \textbf{Gradle}: per la gestione della build di progetto è stato scelto Gradle, per la sua alta velocità;
\item \textbf{Travis CI}: è stato inoltre utilizzato Travis CI come strumento di \textit{continuos integration}: esso permette di verificare, ad ogni nuova modifica del repository, che il progetto continui a compilare con successo e a controllare che i tests passino, in un ambiente unico;
\item \textbf{Trello}: come strumento di tracciamento delle fasi del ciclo di sviluppo. È possibile visionare la \textit{Kanban} al seguente \href{https://trello.com/b/iTJXBZUX}{link}.
\end{itemize}
